% Master Paper Bundle — SU(2) 3nj Series
% Option 1: Comprehensive single paper
% Option 2: Series template (uncomment sections as needed)

\documentclass[12pt,letterpaper]{article}

% Standard packages
\usepackage{amsmath,amssymb,amsthm}
\usepackage{graphicx}
\usepackage{hyperref}
\usepackage{cleveref}

% Shared macros
% shared-macros.tex
% Shared LaTeX macros for SU(2) 3nj series papers
% To be included via: % shared-macros.tex
% Shared LaTeX macros for SU(2) 3nj series papers
% To be included via: % shared-macros.tex
% Shared LaTeX macros for SU(2) 3nj series papers
% To be included via: \input{../shared/shared-macros}

% Mathematical notation
\newcommand{\Rational}{\mathbb{Q}}
\newcommand{\Integer}{\mathbb{Z}}
\newcommand{\Real}{\mathbb{R}}
\newcommand{\Complex}{\mathbb{C}}

% Quantum angular momentum conventions
\newcommand{\ket}[1]{|#1\rangle}
\newcommand{\bra}[1]{\langle#1|}
\newcommand{\braket}[2]{\langle#1|#2\rangle}

% 3nj symbols
\newcommand{\threej}[6]{\begin{pmatrix} #1 & #2 & #3 \\ #4 & #5 & #6 \end{pmatrix}}
\newcommand{\sixj}[6]{\begin{Bmatrix} #1 & #2 & #3 \\ #4 & #5 & #6 \end{Bmatrix}}
\newcommand{\ninej}[9]{\begin{Bmatrix} #1 & #2 & #3 \\ #4 & #5 & #6 \\ #7 & #8 & #9 \end{Bmatrix}}

% Wigner D-matrices
\newcommand{\Dmat}[3]{D^{#1}_{#2,#3}}

% Hypergeometric functions
\newcommand{\hypergeo}[5]{{}_{#1}F_{#2}\!\left[\begin{matrix} #3 \\ #4 \end{matrix}; #5\right]}
\newcommand{\hypergeoF}[3]{{}_{#1}F_{#2}\!\left(#3\right)}

% Pochhammer symbol (rising factorial)
\newcommand{\pochhammer}[2]{(#1)_{#2}}

% Summation shortcuts
\newcommand{\sumk}{\sum_{k}}
\newcommand{\sumt}{\sum_{t}}

% Special cases
\newcommand{\CG}[6]{\langle #1\,#2; #3\,#4 | #5\,#6 \rangle}  % Clebsch-Gordan

% Triangle inequality notation
\newcommand{\triangle}{\Delta}
\newcommand{\tritest}[3]{\triangle(#1,#2,#3)}

% Graph theory (for generating functionals)
\newcommand{\vertex}{\mathcal{V}}
\newcommand{\edge}{\mathcal{E}}
\newcommand{\graph}{\mathcal{G}}

% Recurrence notation
\newcommand{\reccoef}[1]{c_{#1}}   % recurrence coefficients

% Abbreviations
\DeclareMathOperator{\sgn}{sgn}
\DeclareMathOperator{\tr}{tr}
\DeclareMathOperator{\rank}{rank}
\DeclareMathOperator{\cond}{cond}  % condition number

% Theorem-like environments (can be customized per paper)
% \newtheorem{theorem}{Theorem}
% \newtheorem{lemma}[theorem]{Lemma}
% \newtheorem{proposition}[theorem]{Proposition}
% \newtheorem{corollary}[theorem]{Corollary}


% Mathematical notation
\newcommand{\Rational}{\mathbb{Q}}
\newcommand{\Integer}{\mathbb{Z}}
\newcommand{\Real}{\mathbb{R}}
\newcommand{\Complex}{\mathbb{C}}

% Quantum angular momentum conventions
\newcommand{\ket}[1]{|#1\rangle}
\newcommand{\bra}[1]{\langle#1|}
\newcommand{\braket}[2]{\langle#1|#2\rangle}

% 3nj symbols
\newcommand{\threej}[6]{\begin{pmatrix} #1 & #2 & #3 \\ #4 & #5 & #6 \end{pmatrix}}
\newcommand{\sixj}[6]{\begin{Bmatrix} #1 & #2 & #3 \\ #4 & #5 & #6 \end{Bmatrix}}
\newcommand{\ninej}[9]{\begin{Bmatrix} #1 & #2 & #3 \\ #4 & #5 & #6 \\ #7 & #8 & #9 \end{Bmatrix}}

% Wigner D-matrices
\newcommand{\Dmat}[3]{D^{#1}_{#2,#3}}

% Hypergeometric functions
\newcommand{\hypergeo}[5]{{}_{#1}F_{#2}\!\left[\begin{matrix} #3 \\ #4 \end{matrix}; #5\right]}
\newcommand{\hypergeoF}[3]{{}_{#1}F_{#2}\!\left(#3\right)}

% Pochhammer symbol (rising factorial)
\newcommand{\pochhammer}[2]{(#1)_{#2}}

% Summation shortcuts
\newcommand{\sumk}{\sum_{k}}
\newcommand{\sumt}{\sum_{t}}

% Special cases
\newcommand{\CG}[6]{\langle #1\,#2; #3\,#4 | #5\,#6 \rangle}  % Clebsch-Gordan

% Triangle inequality notation
\newcommand{\triangle}{\Delta}
\newcommand{\tritest}[3]{\triangle(#1,#2,#3)}

% Graph theory (for generating functionals)
\newcommand{\vertex}{\mathcal{V}}
\newcommand{\edge}{\mathcal{E}}
\newcommand{\graph}{\mathcal{G}}

% Recurrence notation
\newcommand{\reccoef}[1]{c_{#1}}   % recurrence coefficients

% Abbreviations
\DeclareMathOperator{\sgn}{sgn}
\DeclareMathOperator{\tr}{tr}
\DeclareMathOperator{\rank}{rank}
\DeclareMathOperator{\cond}{cond}  % condition number

% Theorem-like environments (can be customized per paper)
% \newtheorem{theorem}{Theorem}
% \newtheorem{lemma}[theorem]{Lemma}
% \newtheorem{proposition}[theorem]{Proposition}
% \newtheorem{corollary}[theorem]{Corollary}


% Mathematical notation
\newcommand{\Rational}{\mathbb{Q}}
\newcommand{\Integer}{\mathbb{Z}}
\newcommand{\Real}{\mathbb{R}}
\newcommand{\Complex}{\mathbb{C}}

% Quantum angular momentum conventions
\newcommand{\ket}[1]{|#1\rangle}
\newcommand{\bra}[1]{\langle#1|}
\newcommand{\braket}[2]{\langle#1|#2\rangle}

% 3nj symbols
\newcommand{\threej}[6]{\begin{pmatrix} #1 & #2 & #3 \\ #4 & #5 & #6 \end{pmatrix}}
\newcommand{\sixj}[6]{\begin{Bmatrix} #1 & #2 & #3 \\ #4 & #5 & #6 \end{Bmatrix}}
\newcommand{\ninej}[9]{\begin{Bmatrix} #1 & #2 & #3 \\ #4 & #5 & #6 \\ #7 & #8 & #9 \end{Bmatrix}}

% Wigner D-matrices
\newcommand{\Dmat}[3]{D^{#1}_{#2,#3}}

% Hypergeometric functions
\newcommand{\hypergeo}[5]{{}_{#1}F_{#2}\!\left[\begin{matrix} #3 \\ #4 \end{matrix}; #5\right]}
\newcommand{\hypergeoF}[3]{{}_{#1}F_{#2}\!\left(#3\right)}

% Pochhammer symbol (rising factorial)
\newcommand{\pochhammer}[2]{(#1)_{#2}}

% Summation shortcuts
\newcommand{\sumk}{\sum_{k}}
\newcommand{\sumt}{\sum_{t}}

% Special cases
\newcommand{\CG}[6]{\langle #1\,#2; #3\,#4 | #5\,#6 \rangle}  % Clebsch-Gordan

% Triangle inequality notation
\newcommand{\triangle}{\Delta}
\newcommand{\tritest}[3]{\triangle(#1,#2,#3)}

% Graph theory (for generating functionals)
\newcommand{\vertex}{\mathcal{V}}
\newcommand{\edge}{\mathcal{E}}
\newcommand{\graph}{\mathcal{G}}

% Recurrence notation
\newcommand{\reccoef}[1]{c_{#1}}   % recurrence coefficients

% Abbreviations
\DeclareMathOperator{\sgn}{sgn}
\DeclareMathOperator{\tr}{tr}
\DeclareMathOperator{\rank}{rank}
\DeclareMathOperator{\cond}{cond}  % condition number

% Theorem-like environments (can be customized per paper)
% \newtheorem{theorem}{Theorem}
% \newtheorem{lemma}[theorem]{Lemma}
% \newtheorem{proposition}[theorem]{Proposition}
% \newtheorem{corollary}[theorem]{Corollary}


% Theorem environments
\newtheorem{theorem}{Theorem}[section]
\newtheorem{lemma}[theorem]{Lemma}
\newtheorem{proposition}[theorem]{Proposition}
\newtheorem{corollary}[theorem]{Corollary}
\newtheorem{definition}[theorem]{Definition}

\theoremstyle{remark}
\newtheorem{remark}[theorem]{Remark}
\newtheorem{example}[theorem]{Example}

% Title and authors
\title{Unified Closed-Form Representations and Generating Functionals \\
       for SU(2) 3n-j Recoupling Coefficients}

\author{
  % Authors TBD
}

\date{\today}

\begin{document}

\maketitle

\begin{abstract}
We present a unified framework for closed-form representations and generating functionals for SU(2) 3n-j recoupling coefficients (Wigner symbols). This work consolidates:
\begin{itemize}
\item Closed-form hypergeometric product formulas for general 3nj symbols
\item Uniform hypergeometric representations (4F3, etc.)
\item Finite three-term recurrence relations with stability analysis
\item Universal generating functionals for arbitrary-valence coupling graphs
\item Matrix elements for multi-valence nodes via functional derivatives
\end{itemize}

All representations are validated against SymPy's Wigner functions, with comprehensive cross-verification across independent implementations. Reference datasets, stability reports, and reproducible validation scripts are provided.

\noindent \textbf{Keywords:} Wigner symbols, 3nj coefficients, recoupling, hypergeometric functions, generating functionals, quantum angular momentum
\end{abstract}

\tableofcontents

\section{Introduction}
\label{sec:intro}

% Motivation: quantum angular momentum coupling, applications in atomic/molecular/nuclear physics
% Historical context: Racah, Wigner, etc.
% Problem statement: need for efficient, numerically stable representations
% Our contributions: unified framework with validation

\section{Background and Notation}
\label{sec:background}

\subsection{SU(2) Representation Theory}

The SU(2) Lie group describes rotations in three-dimensional space and quantum angular momentum. Irreducible representations are labeled by half-integer spins $j \in \{0, 1/2, 1, 3/2, \ldots\}$. Coupling two representations $j_1 \otimes j_2$ decomposes into a direct sum weighted by Clebsch-Gordan coefficients (3j symbols):
$$
\begin{pmatrix} j_1 & j_2 & j_3 \\ m_1 & m_2 & m_3 \end{pmatrix}
$$

Higher-order recoupling (6j, 9j, 12j, $\ldots$ symbols) arises when coupling multiple angular momenta. The 6j symbol describes the transformation between two coupling schemes for three angular momenta:
$$
\left\{ \begin{matrix} j_1 & j_2 & j_3 \\ j_4 & j_5 & j_6 \end{matrix} \right\}
$$

Selection rules include triangle inequalities (e.g., $|j_1-j_2| \leq j_3 \leq j_1+j_2$) and parity constraints ($j_1+j_2+j_3 \in \mathbb{Z}$).

\subsection{Historical Context and Existing Approaches}

The theory of angular momentum recoupling has a rich seventy-year history. Wigner~\cite{wigner1931gruppentheorie} established the foundation with 3j symbols, extended by Racah~\cite{racah1942theory} to 6j coefficients. The comprehensive treatise by Varshalovich et al.~\cite{varshalovich1988quantum} systematized computational methods and remains the standard reference.

Prior computational approaches include:

\paragraph{Summation formulas} Direct evaluation of Racah sums. Accurate but computationally expensive for large quantum numbers. Schulten and Gordon~\cite{schulten1975exact} developed exact recursive evaluation.

\paragraph{Hypergeometric representations} Recognition that 3nj symbols are special hypergeometric functions. Raynal~\cite{raynal1979complete} provided complete 6j representations. We extend this to universal product formulas.

\paragraph{Recurrence relations} Three-term recurrences build values from boundaries. Luscombe and Luban~\cite{luscombe1998three} systematized 3j/6j recurrences. We derive closed forms for coefficients.

\paragraph{Generating functionals} Yutsis et al.~\cite{yutsis1962mathematical} introduced graphical methods. We develop systematic generating functional calculus for arbitrary valence.

\paragraph{Symmetry relations} Regge~\cite{regge1958symmetry,regge1959symmetry} discovered profound permutation symmetries. Our formulas naturally encode these through hypergeometric structure.

Modern implementations (SymPy~\cite{meurer2017sympy}, specialized libraries~\cite{johansson2016fast,rasch2003efficient}) combine these with numerical optimization.

\subsection{Applications}

3nj symbols appear across quantum physics:
\begin{itemize}
\item \textbf{Atomic/molecular}: Multi-electron systems, hyperfine structure
\item \textbf{Nuclear structure}: Shell models, coupled-cluster methods
\item \textbf{Quantum gravity}: Spin networks~\cite{rovelli1995spin}, loop quantum gravity recoupling~\cite{depietri1996regge}
\item \textbf{Quantum information}: Entanglement, tensor networks
\end{itemize}

\subsection{Novel Contributions}

We advance beyond prior work through:
\begin{enumerate}
\item \textbf{Universal closed forms}: Hypergeometric product formulas unifying special cases
\item \textbf{Generating functionals}: Systematic framework for arbitrary-valence nodes
\item \textbf{Rigorous validation}: Cross-verification of three implementations + 50-digit reference data
\item \textbf{Stability analysis}: Documented failure modes and UQ protocols
\end{enumerate}

\section{Closed-Form Hypergeometric Formulas}
\label{sec:closed-form}

% Import content from:
% su2-3nj-closedform/A Closed-Form Hypergeometric Product Formula...tex

\subsection{Product Formula for General 3nj}
% Main theorem + proof sketch

\subsection{Computational Complexity}
% Comparison: product vs summation

\section{Uniform Hypergeometric Representation}
\label{sec:uniform}

% Import content from:
% su2-3nj-uniform-closed-form/Universal Closed-Form Hypergeometric...tex

\subsection{4F3 Representation for 6j Symbols}
% Explicit formula

\subsection{Extension to 9j Symbols}
% Generalization strategy

\section{Finite Recurrence Relations}
\label{sec:recurrences}

% Import content from:
% su2-3nj-recurrences/Closed-Form Finite Recurrences...tex

\subsection{Three-Term Recurrences}
% Explicit coefficient formulas

\subsection{Stability Analysis}
% Forward vs backward recursion
% Condition numbers and error propagation

\section{Generating Functionals}
\label{sec:generating-functional}

% Import content from:
% su2-3nj-generating-functional/A Universal Generating Functional...tex

\subsection{Graph-Theoretic Formulation}
% Coupling graphs, vertex/edge notation

\subsection{Determinant Representation}
% $G(x_e) = 1/\det(I-K(x_e))$

\subsection{Series Expansion}
% Coefficients as 3nj symbols

\section{Arbitrary-Valence Node Matrix Elements}
\label{sec:node-matrix}

% Import content from:
% su2-node-matrix-elements/Closed-Form Matrix Elements...tex

\subsection{Functional Derivative Approach}
% $M_v = \partial^k G / \partial s_1 \cdots \partial s_k |_{s=0}$

\subsection{Determinant Stability}
% Regularization, condition numbers

\section{Validation and Cross-Verification}
\label{sec:validation}

We validate our closed-form representations and computational implementations through multiple independent verification routes.

\subsection{Test Methodology}

Our validation harness employs:
\begin{itemize}
\item \textbf{pytest framework}: 161 tests across five independent implementations
\item \textbf{SymPy cross-checks}: Exact symbolic comparison against \texttt{wigner\_6j} and \texttt{wigner\_9j}
\item \textbf{Deterministic reference datasets}: High-precision (50 decimal places) golden values generated with \texttt{mpmath}
\item \textbf{Cross-implementation verification}: Three independent computational routes (closed-form, uniform representation, generating functional)
\end{itemize}

All tests are reproducibly executed via a single command (\texttt{python -m pytest}) and produce JSON validation reports for inclusion in this paper.

\subsection{Cross-Implementation Comparison}

Table~\ref{tab:cross-verification} demonstrates exact agreement between all three independent implementations and SymPy across five diverse spin configurations. The test cases span:
\begin{itemize}
\item Uniform integer spins
\item Sequential integer configurations
\item Uniform higher spins ($j=2$)
\item Half-integer spins ($j=1/2, 3/2$)
\item Mixed integer and half-integer cases
\end{itemize}

All implementations agree to machine precision, validating both the mathematical derivations and the computational implementations.

\subsection{Higher-n Reference Data}

To establish deterministic validation baselines beyond 6j symbols, we computed high-precision (50 decimal places) reference values for 9j symbols using \texttt{mpmath}. Table~\ref{tab:9j-reference} shows seven representative cases spanning:
\begin{itemize}
\item Uniform configurations ($j=1, j=2$)
\item Sequential ladder patterns
\item Half-integer basic cases
\item Mixed spin configurations
\item Edge cases (all zeros, partial zeros with triangle violations)
\end{itemize}

All 7 cases computed successfully with documented stability behavior. Table~\ref{tab:validation-summary} provides a comprehensive summary of test coverage across all validation suites.

% Include auto-generated validation tables
% Validation tables auto-generated from JSON reports
% Generated by scripts/generate_validation_tables.py

% Auto-generated from data/integration_validation_report.json
\begin{table}[htbp]
\centering
\caption{Cross-Implementation Verification Results}
\label{tab:cross-verification}
\begin{tabular}{lcccc}
\hline
Configuration & Spins & SymPy & Gen.~Func. & Closed Form \\
\hline
uniform integer & (1, 1, 1, 1, 1, 1) & $\frac{1}{6}$ & $\frac{1}{6}$ & $\frac{1}{6}$ \\
sequential integer & (1, 2, 3, 4, 5, 6) & $\frac{\sqrt{1430}}{2145}$ & $\frac{\sqrt{1430}}{2145}$ & $\frac{\sqrt{1430}}{2145}$ \\
uniform j=2 & (2, 2, 2, 2, 2, 2) & $- \frac{3}{70}$ & $- \frac{3}{70}$ & $- \frac{3}{70}$ \\
half-integer & (1/2, 1/2, 1, 1/2, 1/2, 1) & $\frac{1}{6}$ & $\frac{1}{6}$ & $\frac{1}{6}$ \\
mixed & (1, 1/2, 3/2, 1, 1/2, 3/2) & $- \frac{1}{12}$ & $- \frac{1}{12}$ & $- \frac{1}{12}$ \\
\hline
\multicolumn{5}{l}{All implementations agree to machine precision.} \\
\hline
\end{tabular}
\end{table}

% Auto-generated from data/higher_n_reference_9j.json
\begin{table}[htbp]
\centering
\caption{High-Precision 9j Symbol Reference Dataset (50 decimal places)}
\label{tab:9j-reference}
\begin{tabular}{lcc}
\hline
Configuration & Exact Value & Status \\
\hline
uniform j=1 & $0$ & \checkmark \\
uniform j=2 & $\frac{41}{2450}$ & \checkmark \\
sequential ladder & $- \frac{61}{66150}$ & \checkmark \\
half-integer basic & $- \frac{1}{18}$ & \checkmark \\
mixed spin configuration & $- \frac{1}{48}$ & \checkmark \\
all zeros & $1$ & \checkmark \\
partial zeros & $\frac{1}{3}$ & \checkmark \\
\hline
\multicolumn{3}{l}{Precision: 50 decimal places} \\
\hline
\end{tabular}
\end{table}

% Auto-generated from data/higher_n_reference_12j.json
\begin{table}[htbp]
\centering
\caption{High-Precision 12j Symbol Reference Dataset (50 decimal places)}
\label{tab:12j-reference}
\begin{tabular}{lcc}
\hline
Configuration & Exact Value & Status \\
\hline
uniform j=1 & $\frac{1}{54}$ & \checkmark \\
uniform j=1/2 & $0$ & \checkmark \\
four triangular triads & $\frac{1}{27}$ & \checkmark \\
all j=1 & $\frac{1}{54}$ & \checkmark \\
all zeros & $1$ & \checkmark \\
alternating 1,2 & $\frac{\sqrt{5}}{30000} + \frac{7}{6000}$ & \checkmark \\
\hline
\multicolumn{3}{l}{Method: 6j\_symbol\_decomposition, Precision: 50 dps} \\
\hline
\end{tabular}
\end{table}

% Summary of validation coverage
\begin{table}[htbp]
\centering
\caption{Validation Test Coverage Summary}
\label{tab:validation-summary}
\begin{tabular}{lcc}
\hline
Test Suite & Tests & Pass Rate \\
\hline
Hub integration (all checks) & 21 & 100\% \\
generating-functional unit tests & 43 & 100\% \\
uniform-closed-form unit tests & 45 & 100\% \\
closedform unit tests & 27 & 100\% \\
recurrences unit tests & 18 & 100\% \\
node-matrix-elements unit tests & 24 & 100\% \\
\hline
\textbf{Total} & \textbf{178} & \textbf{100\%} \\
\hline
\end{tabular}
\end{table}

\subsection{Stability Regimes}
% Documented failure modes
% Numerical precision requirements

\section{Computational Performance}
\label{sec:performance}

% Benchmarks from various repos
% Comparison of different representations

\section{Applications}
\label{sec:applications}

% Atomic/molecular spectroscopy
% Nuclear structure
% Quantum information

\section{Conclusions}
\label{sec:conclusions}

% Summary of contributions
% Future work: extensions to other groups, higher symbols

\section*{Acknowledgments}

% TBD

\bibliographystyle{plain}
\bibliography{../shared/shared-bibliography}

\appendix

\section{Proof Details}
\label{app:proofs}

% Extended proofs omitted from main text

\section{Reference Datasets}
\label{app:datasets}

% Pointer to JSON files in repository

\section{Software Implementation}
\label{app:software}

% Links to GitHub repos
% Installation instructions
% Example usage

\end{document}
